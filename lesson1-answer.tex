% EXERCISE ONE

\documentclass{amsart}              
\thispagestyle{empty}      

\begin{document}

\begin{center}
\textbf{Exercise One: The Basics}
\end{center}
  
A certain math professor will \textbf{scream} and may even \textbf{cry} if you confuse 
the Latin terms \emph{id est}, meaning ``that is to say'' and \emph{exempli
gratia}, meaning ``for instance.'' Compare the following.\\

I adore polynomials, \emph{e.g.}, $x^4+x^2+1$.

I adore polynomials, \emph{i.e.}, expressions of the form
$a_0+a_1x+a_2x^2+\dots+a_nx^n$, where $a_0, a_1, ..., a_n$ are constants and $n$ is a
non-negative integer.\\

Do you lie awake at night wondering what is the smallest positive integer that can be
written as the sum of two perfect cubes in two distinct ways? Well, wonder no more:
$$1729=10^3+9^3=12^3+1^3$$

We all learned in first grade that $(a+b)^3\neq a^3+b^3$, but when we study modular
arithmetic we'll find that the two \textit{are} equal in ``mod 3.''\\

The famous mathematician Euler (rhymes with ``boiler'' \textbf{NOT} with ``ruler'') used
the geometric series formula $a+ar+ar^2+ar^3+...=a/(1-r)$ to
conclude the following.
$$1-1+1-1+1-\dots = \frac{1}{2}$$
But we know the formula only applies if $-1<r<1$, so this result is not valid.

\end{document}
